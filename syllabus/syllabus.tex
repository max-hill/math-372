\documentclass[10pt]{article}
% Math Packages
\usepackage{amsmath, mathtools}
\usepackage{amssymb}
\usepackage{amsthm}
\usepackage{amsfonts}
\usepackage{bbm}
\usepackage{breqn}
\usepackage[margin=1in]{geometry}
\usepackage{graphicx}
\usepackage{tikz}
\usepackage{forest}
\usepackage{tikz-qtree}
\graphicspath{ {./images/} }
\usepackage{hyperref}
\usepackage[capitalize]{cleveref}
\usepackage[shortlabels]{enumitem}
\usetikzlibrary{arrows,matrix,positioning}

% Colorful Notes
\usepackage{color} \definecolor{Red}{rgb}{1,0,0} \definecolor{Blue}{rgb}{0,0,1}
\definecolor{Purple}{rgb}{.5,0,.5} \def\red{\color{Red}} \def\blue{\color{Blue}}
\def\gray{\color{gray}} \def\purple{\color{Purple}}
\newcommand{\rnote}[1]{{\red [#1]}} % \rnote{foo} gives '[foo]' in red
\newcommand{\pnote}[1]{{\purple [#1]}} % \pnote{foo} gives '[foo]' in purple
\newcommand{\bnote}[1]{{\blue #1}} % \bnote{foo} gives 'foo' in blue
\newcommand{\gnote}[1]{{\gray #1}} % \gnote{foo} gives 'foo' in gray
\newcommand{\Max}[1]{{\purple [#1]}} % \bnote{foo} then 'foo' is blue
\definecolor{RoyalBlue}{cmyk}{1, 0.50, 0, 0}
\newcommand{\dempfcolor}[1]{{\color{RoyalBlue}#1}} 
\newcommand{\demph}[1]{\dempfcolor{{\sl #1}}}




% Math Environments
\newtheorem{theorem}{Theorem}
\newtheorem{definition}[theorem]{Definition}
\newtheorem{assumption}[theorem]{Assumption}
\newtheorem{lemma}[theorem]{Lemma}
\newtheorem{remark}[theorem]{Remark}
\newtheorem{proposition}[theorem]{Proposition}
\newtheorem{corollary}[theorem]{Corollary}
\newtheorem{example}[theorem]{Example} 
\newtheorem{question}[theorem]{Question}

% Custom Math Commands
\newcommand{\vt}{\vskip 5mm} % vertical space
\newcommand{\fl}{\\\noindent\textbf} % first line
\newcommand{\Fl}{\vt\noindent\textbf} % first line with space above


\begin{document}
\begin{center}
  \section*{Syllabus for Math 372: Elementary Probability and Statistics}
  \subsection*{Spring 2026}
\end{center}

\Fl{Instructor:} Max Hill
\fl{Office:} Physical Sciences Building 304
\fl{Office Hours:} TBD (or other times by appointment).
\fl{Course website:} \url{https://max-hill.github.io/math-372/}
\Fl{Place and time:} Web 113 at 12:30-1:20pm (MWF)

\Fl{Textbook:} \textit{Probability and Statistics for Engineering and the Sciences} by Jay L.~Devore (9th
edition). ISBN-13: \texttt{978-1-305-25180-9}

\Fl{Course prerequistes:} Math 216 or Math 242 or Math 252A. If you don't meet the prerequisites, you'll
need to get approval from the math department office.

\Fl{Exams:} There will be two midterms and a final exam. The final will be cumulative, but will emphasize
material after the midterm.  The exam dates are as follows:
\begin{center} \begin{minipage}{5.0in}
    \begin{flushleft}
      Midterm 1 \dotfill~Wednesday, Feb 18 (in class) \\
      Midterm 2 \dotfill~Friday, March 27 (in class) \\
      Final exam \dotfill~Friday, May 15 at 12:00-2:00pm\\
    \end{flushleft}
  \end{minipage}
\end{center}
% If you do better on the final than the midterm, then your grade on the final will replace your midterm
% grade.


\Fl{Grading:} Homework  (20\%), Midterms (40\%), Final (40\%). The following grade cutoffs will be used at
the end of the semester to determine final grades:
\begin{center}
  \begin{tabular}{|c|c|c|c|c|c|c|c|c|c|c|}
    \hline
    D&D+&C-&C&C+&B-&B&B+&A-&A&A+  \\
    \hline
    60\%&67\%&70\%&73\%&77\%&80\%&83\%&87\%&90\%&93\%&97\%\\
    \hline
  \end{tabular}
\end{center}
% I may change the above numbers, but if I do so it will be in your favor.


\Fl{Homework:} Homework will be a mix of worksheets/short in-class quizzes and
written assignments.
\begin{itemize}
  \item You have one free `no questions asked' homework extension.
  %\item Your lowest homework score will be dropped at the end of the semester.
  \item You may collaborate with classmates on the homeworks. But if you do so, you must (1) make an
  effort write up your solutions on your own, using your own words, and (2) list the names of the people
  who you worked with.
\end{itemize}

\Fl{Make-up policy:} Make-up exams are allowed only in three types of circumstances: (1) in accordance
with university policies, such as conflict with a religious observation, (2) conflicts with another
university-related event, or (3) exceptional circumstances, such as a last-minute medical or family
emergency with verification. In the first two cases, notice must be given to the instructor two weeks in
advance.


\Fl{Incompletes:} An incomplete is possible only if all of the following
apply: (1) you have a compelling personal reason, e.g., serious illness or
accident (a proof, e.g., report from a doctor or police must be shown); (2)
your work so far would receive a passing grade; and (3) there is a good chance
you will complete the course with a passing grade within the allotted time.
Thus, expecting to fail the class is not a reason to ask for an incomplete.

\Fl{Accommodations:} Any student who feels s/he may need an accommodation based on the impact of a
disability is invited to contact me privately. I would be happy to work with you, and the KOKUA Program
(Office for Students with Disabilities) to ensure reasonable accommodations in my course. KOKUA can be
reached at (808) 956-7511 or (808) 956-7612 (voice/text) in Room 013 of the Queen Lili`uokalani Center
for Student Services.



\newpage
\Fl{Tentative course outline:} This course is a problem-oriented introduction to the basic concepts of
probability and statistics, providing a foundation for applications and further study.

\begin{itemize}
  \item \textbf{Weeks 1-2:} Introduction to probability theory 
  \begin{itemize}
    \item Experiments, events, sets, probabilities, random variables. Equally likely outcomes, counting
    techniques. Conditional probability. Independence. Bayes' theorem. (Sections: 2.1-2.5)
  \end{itemize}
  \item \textbf{Weeks 3-5:} Random variables
  \begin{itemize}
    \item Discrete random variables (1.5 weeks): Expected values, mean, variance, binomial distribution,
    Poisson distribution. Moment generating functions.  (Sections: 3.1-3.6)
    \item Continuous Random variables (1.5 weeks): Uniform, exponential, gamma, and normal distributions.
    Intuitive treatment of the Poisson process and development of the relationship with gamma
    distributions. (Sections: 4.1-4.4)
  \end{itemize}
  \item \textbf{Midterm 1 (Feb 18)}
  \item \textbf{Weeks 6-7:} Multivariate distributions
  \begin{itemize}
    \item Calculation of probability, covariance, correlation, marginals, conditions. Distributions of
    sums of random variables and sampling distributions. Central limit theorem. (Sections: 1.1, 1.3, 1.4, 5.1-5.7)
  \end{itemize}
  \item \textbf{Week 8:} Catch-up, review.
  \item \textbf{Week 9:} Introduction to statistical estimation
  \begin{itemize}
    \item Point and confidence interval estimation. Maximum likelihood, optimal, and unbiased estimators.
    Examples. (Sections 6.1, 6.2)
  \end{itemize}
  \item \textbf{Midterm 2 (March 27)}
  \item \textbf{Weeks 10-12:} Large sample inference
  \begin{itemize}
    \item Estimation (1.5 weeks): Types and comparison of estimators; sampling distributions
    for means/proportions, and their use in large sample estimation; sample size. (Sections 7.1,
    7.2)
    \item Hypothesis testing (1.5 weeks): Components of a test; signficance and power; p-values;
    large-sample tests for means and proportions (Sections: 8.1-8.4)
  \end{itemize}
  \item \textbf{Week 13:} Small sample inference
  \begin{itemize}
    \item t-distribution, with applications to small sample estimation and testing; $\chi^{2}$ and $F$
    distributions, with applications to inference about variances (Sections: 7.3, 7.4, 8.3)
  \end{itemize}
  \item \textbf{Weeks 14-16:} Regression and $\chi^{2}$ tests
  \begin{itemize}
    \item Regression (1.5 weeks): Least squares, correlation coefficient, inference (Sections:
    12.1-12.5)
    \item $\chi^{2}$ tests: multinomial dsitributions, contingency tables, goodness-of-fit (Sections: 14.1-14.3)
  \end{itemize}
  \item \textbf{Last day of instruction: May 6}
  \item \textbf{Final Exam: May 15}
\end{itemize} 
\end{document}