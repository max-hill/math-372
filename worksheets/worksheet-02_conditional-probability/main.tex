 \documentclass[10pt]{amsart}
\usepackage{graphicx,amsmath,amssymb,amsfonts,mathrsfs,latexsym}
\usepackage[top=2cm,bottom=1.5cm,left=2cm,right=2cm]{geometry}
\usepackage{caption}
\usepackage{epstopdf}
\pagestyle{empty}
\theoremstyle{definition}
\newtheorem{prob}{Problem}[section]
\newcounter{PROB}
\newcounter{PN}[PROB]
\newcounter{QN}[PROB]
\setcounter{QN}{0}
\setcounter{PN}{0}
\setcounter{PROB}{0}
\newcommand{\Pnum}[1][]{\begin{center}\stepcounter{PROB}{\large\textbf{Problem \arabic{PROB}: #1}}\end{center}\par}
\newcommand{\pnum}[1][]{\stepcounter{PN}{\large \textbf{Part \arabic{PN}: #1}}\newline\par}
\newcommand{\qnumn}{\stepcounter{QN}\textbf{Question \arabic{PROB}.\arabic{QN}: }}
\newcommand{\qnum}[1][]{\stepcounter{QN}\par\textbf{Question \arabic{PROB}.\arabic{QN}: }(#1 points) }
\newcommand{\answerboxn}[1][]{\phantom{.}\hfill\framebox[5cm]{\begin{minipage}{1px}\hfill\vspace{.4in}\end{minipage}\hfill#1\ }\newline\newline}
\newcommand{\answerbox}[1][]{\\\phantom{.}\hfill\framebox[5cm]{\begin{minipage}{1px}\hfill\vspace{.4in}\end{minipage}\hfill#1\ }\newline\newline}

% Answer box for Xantharian Squares (Algebra.tex)
\newcommand{\xanthariananswerbox}[1][\phantom{E}]{\framebox[2cm]{\begin{minipage}{.4in}\vspace{.1in}\begin{flushright}\hfill#1\hfill\end{flushright} \vfill \vspace{.1in}\end{minipage}\hfill}}

% Answer box for dice-games.tex
\newcommand{\dicegamesanswerbox}[1][\phantom{E}]{\begin{flushright}\framebox[2cm]{\begin{minipage}{.3in}\vspace{.07in}\begin{flushright}\hfill#1\hfill\end{flushright} \vfill \vspace{.16in}\end{minipage}\hfill}\end{flushright}}

\usepackage[shortlabels]{enumitem} % For Xanthar 
\newtheorem{problem}{Problem} % For Xanthar
\usepackage{multirow} % needed for the tables in dice-games

% For drawing boxes around stuff
\usepackage[linewidth=1pt]{mdframed}

%%%Tikz stuff
\usepackage{pgf,tikz,pgfplots}
\pgfplotsset{compat=1.15}
\usepackage{mathrsfs}
\usetikzlibrary{arrows}
\usetikzlibrary{math, calc}

%%% To comment out stuff
\usepackage{verbatim}

%% Columns
\usepackage{multicol}

% Floating figures
\usepackage{float}


\usepackage{skull}
\def\P{\mathbb{P}} % Probability


\newcommand{\dnum}[1][]{\stepcounter{PN}{\large \textbf{Part \arabic{PN}: #1}}\vspace{.5em}\par}


\begin{document}



\section*{Worksheet 2: Conditional probability}



\begin{problem}
  We return to the dice-rolling pirate, Diego. Suppose Diego rolls two dice and then adds the numbers
  together. The possible outcomes are $2,3,4,5,6,7,8,9,10,11, \text{ and }12$, but recall that these
  numbers are not all equally likely. 
  \begin{enumerate}[(a)]
    \item What is the probability that the sum of the two dice is greater than 4? 
    \vspace{.75cm} %\dicegamesanswerbox
    \item Given that the sum of the dice is greater than 4, what is the probability that Diego's dice
    added up to 7?
    \vspace{.75cm} %\dicegamesanswerbox
    \item What is the probability that the dice add up to an even number?
    \vspace{.75cm} %\dicegamesanswerbox
    \item What is the probability that the dice add up to an odd number?
    \vspace{.75cm} %\dicegamesanswerbox
    \item Given that the dice add up to an even number, what is the probability that Diego rolled at
    least one 4?
    \vspace{.75cm} %\dicegamesanswerbox
    \item Given that Diego's dice sum to 7, what is the probability that he rolled a 3 or a 5?
    \vspace{.75cm} %\dicegamesanswerbox
    \item What is the probability that at least one of  Diego's dice was a 3?
    \vspace{.75cm} %\dicegamesanswerbox
    \item Given that at least one of Diego's dice was a 3, what is the probability that his dice add up
    to 8 or 9? \vspace{.75cm} %\dicegamesanswerbox
    \item What is the probability that Diego gets a sum of 4 or 7?
    \vspace{.75cm} %\dicegamesanswerbox
    \item Suppose Diego keeps rolling the two dice until he gets a sum of either 4 or a 7, at which
    point he stops. What's the probability that his last roll was 4?
    \vspace{.75cm} %\dicegamesanswerbox
  \end{enumerate}
\end{problem}

\begin{problem}
  In poker, a \textbf{full house} occurs when you get a 3-of-a-kind and a 2-of-a-kind (with a 5-card hand).

  \begin{enumerate}[(a)]
    \item How many possible full houses are there?
    \vspace{.75cm} %\dicegamesanswerbox
    \item What's the probability of getting a full house?
    \vspace{.75cm} %\dicegamesanswerbox
  \end{enumerate}

  % The number of possible full houses
  % \begin{equation*}
  %   {13\choose 1}\times {4\choose 3}\times {12\choose 1}\times {4\choose 2} = 3,744
  % \end{equation*}

  % The probability of getting a full house is
  % \begin{equation*}
  %   \P\left[\text{full house} \right] = \frac{3,744}{2,598,960} = 0.00144
  % \end{equation*}
\end{problem}

\begin{problem}
  A vampire goes to the blood bank looking to find some type O+ blood (the most delicious type). He finds
  four unlabeled bags of blood. Only one of the bags is O+, but the vampire doesn't know which one, so he
  resorts to taste-testing to find the O+ bag.

  \begin{enumerate}[(a)]
    \item  What is the probability that he must test at
    least 3 bags to find the desired type?
    \vspace{.75cm} %\dicegamesanswerbox
    % \noindent \textit{Solution:} Let $X$ be the number of tested bags. We want to find $\P\left[X \geq
    %   3\right]$.
    % Let $A= \left[ \text{first bag isn't O+} \right]$. Let $B = \left[ \text{second bag isn't O+} \right]$
    % \begin{align*}
    %   \P\left[ X \geq 3\right] 
    %   &= \P\left[A\cap B \right]\\
    %   &= \P\left[A \right] \P\left[B\mid A \right]\\
    %   &= \frac{3}{4}\cdot \frac{2}{3}\\
    %   &= \frac{1}{2}.
    % \end{align*}

    \item What's the probability that he must test exactly $3$ bags?
    \vspace{.75cm} %\dicegamesanswerbox
    % \noindent \textit{Solution:} We want $\P\left[X=3 \right]=\P\left[\text{third bag is O+} \right]$.
    % \begin{align*}
    %   \P\left[\text{third bag is O+} \right] 
    %   &= \P\left[\text{third bag is}\mid  \text{first isn't} \cap
    %     \text{second isn't}  \right] \times \P\left[\text{second isn't}\mid \text{first isn't} \right] \cdot
    %     \P\left[\text{first isn't} \right]\\
    %   &= \frac{1}{2}\cdot \frac{2}{3}\cdot \frac{3}{4}\\
    %   &= \frac{1}{4}
    % \end{align*}
  \end{enumerate}
\end{problem}


\end{document}