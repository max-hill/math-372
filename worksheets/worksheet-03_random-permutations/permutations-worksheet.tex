
%%% THIS DOCUMENT NEEDS TO BE COMPILED WITH LUATEX OR LUALATEX ENGINE

\documentclass[10pt]{amsart}
% Math Packages
\usepackage{amsmath, mathtools}
\usepackage{amssymb}
\usepackage{amsthm}
\usepackage{amsfonts}
\usepackage{bbm}
\usepackage{breqn}
\usepackage[margin=1in]{geometry}
\usepackage{graphicx}
\usepackage{tikz}
\usetikzlibrary{arrows.meta}
\usetikzlibrary{calc}
\usepackage{forest}
\usepackage{tikz-qtree}
\graphicspath{ {./images/} }
\usepackage{hyperref}
\usepackage[capitalize]{cleveref}
\usepackage[shortlabels]{enumitem}
\usetikzlibrary{arrows,matrix,positioning}
\usepackage{multicol}

% for the pipe symbol
\usepackage[T1]{fontenc}

% Citing theorems by name. (source: https://tex.stackexchange.com/questions/109843/cleveref-and-named-theorems)
\makeatletter
\newcommand{\ncref}[1]{\cref{#1}\mynameref{#1}{\csname r@#1\endcsname}}

\def\mynameref#1#2{%
  \begingroup
  \edef\@mytxt{#2}%
  \edef\@mytst{\expandafter\@thirdoffive\@mytxt}%
  \ifx\@mytst\empty\else
  \space(\nameref{#1})\fi
  \endgroup
}
\makeatother

% Colorful Notes
\usepackage{color} \definecolor{Red}{rgb}{1,0,0} \definecolor{Blue}{rgb}{0,0,1}
\definecolor{Purple}{rgb}{.5,0,.5} \def\red{\color{Red}} \def\blue{\color{Blue}}
\def\gray{\color{gray}} \def\purple{\color{Purple}}
\newcommand{\rnote}[1]{{\red [#1]}} % \rnote{foo} gives '[foo]' in red
\newcommand{\pnote}[1]{{\purple [#1]}} % \pnote{foo} gives '[foo]' in purple
\newcommand{\bnote}[1]{{\blue #1}} % \bnote{foo} gives 'foo' in blue
\newcommand{\gnote}[1]{{\gray #1}} % \gnote{foo} gives 'foo' in gray
\newcommand{\Max}[1]{{\purple [#1]}} % \bnote{foo} then 'foo' is blue


% Claim numbering (the counter restarts after each proof environment)
\newcounter{claimcount}
\setcounter{claimcount}{0}
\newenvironment{claim}{\refstepcounter{claimcount}\par\addvspace{\medskipamount}\noindent\textbf{Claim \arabic{claimcount}:}}{}
\usepackage{etoolbox}
\AtBeginEnvironment{proof}{\setcounter{claimcount}{0}}
\newenvironment{claimproof}{\par\addvspace{\medskipamount}\noindent\textit{Proof of Claim  \arabic{claimcount}.}}{\hfill\ensuremath{\qedsymbol} \tiny{Claim}

  \medskip}
% Add claim support to cleverref
\crefname{claimcount}{Claim}{Claims}


% Math Environments
\newtheorem{theorem}{Theorem}
\newtheorem{assumption}[theorem]{Assumption}
\newtheorem{lemma}[theorem]{Lemma}
\newtheorem{proposition}[theorem]{Proposition}
\newtheorem{corollary}[theorem]{Corollary}
\newtheorem{question}[theorem]{Question}
\theoremstyle{definition}
\newtheorem{definition}[theorem]{Definition}
\newtheorem{remark}[theorem]{Remark}
\newtheorem{example}[theorem]{Example}
\newtheorem{notation}[theorem]{Notation}
\newtheorem{problem}[theorem]{Problem}

% Matrices and Column Vectors. 
\usepackage{stackengine}
\setstackgap{L}{1.0\normalbaselineskip}
\usepackage{tabstackengine}
\setstackEOL{;}% row separator
\setstackTAB{,}% column separator
\setstacktabbedgap{1ex}% inter-column gap
\setstackgap{L}{1.5\normalbaselineskip}% inter-row baselineskip
\let\nmatrix\bracketMatrixstack  %Usage: \nmatrix{1,2,3\4,5,6}
\newcommand\cv[1]{\setstackEOL{,}\bracketMatrixstack{#1}} %usage: \cv{1,2,3}

% table alignment
\usepackage{array}
\usepackage{booktabs}

% Custom Math Coqmmands
\newcommand{\vt}{\vskip 5mm} % vertical space
\newcommand{\fl}{\noindent\textbf} % first line
\newcommand{\Fl}{\vt\noindent\textbf} % first line with space above
\newcommand{\norm}[1]{\left\lVert#1\right\rVert} % norm
\newcommand{\pnorm}[1]{\left\lVert#1\right\rVert_p} % p-norm
\newcommand{\qnorm}[1]{\left\lVert#1\right\rVert_q} % q-norm
\newcommand{\1}[1]{\textbf{1}_{\left[#1\right]}} % indicator function 
\def\limn{\lim_{n\to\infty}} % shortcut for lim as n-> infinity
\def\sumn{\sum_{n=1}^{\infty}} % shortcut for sum from n=1 to infinity
\def\sumkn{\sum_{k=1}^{n}} % shortcut for sum from k=1 to n
\def\sumin{\sum_{i=1}^{n}} % shortcut for sum from i=1 to n
\def\SAs{\sigma\text{-algebras}} % shortcut for $\sigma$-algebras
\def\SA{\sigma\text{-algebra}} % shortcut for $\sigma$-algebra
\def\Ft{\mathcal{F}_t} % time-indexed sigma-algebra (t)
\def\Fs{\mathcal{F}_s} % time-indexed sigma-algebra (s)
\def\F{\mathcal{F}} % sigma-algebra
\def\G{\mathcal{G}} % sigma-algebra
\def\R{\mathbb{R}} % Real numbers
\def\N{\mathbb{N}} % Natural numbers
\def\Z{\mathbb{Z}} % Integers
\def\E{\mathbb{E}} % Expectation
\def\P{\mathbb{P}} % Probability
\def\Q{\mathbb{Q}} % Q probability
\def\dist{\text{dist}} %Text 'dist' for things like 'dist(x,y)'
\newcommand{\indep}{\perp \!\!\! \perp}  %independence symbol
\def\Var{\mathrm{Var}} % Variance
\def\tr{\mathrm{tr}} % trace

% Brackets and Parentheses
\def\[{\left [}
    \def\]{\right ]}
% \def\({\left (}
%   \def\){\right )}



\usepackage{color}
\definecolor{Red}{rgb}{1,0,0}
\definecolor{Blue}{rgb}{0,0,1}
\definecolor{Purple}{rgb}{.5,0,.5}
\def\red{\color{Red}}
\def\blue{\color{Blue}}
\def\gray{\color{gray}}
\def\purple{\color{Purple}}
\definecolor{RoyalBlue}{cmyk}{1, 0.50, 0, 0}
\newcommand{\dempfcolor}[1]{{\color{RoyalBlue}#1}} 
\newcommand{\demph}[1]{\dempfcolor{{\sl #1}}}

% comment exactly one of the following line to show / hide the solutions
% \newcommand{\solution}[1]{{\purple #1}} % uncomment to show the solutions
\newcommand{\solution}[1]{} % uncomment to hide the solutions

\usepackage{pst-poker}


\begin{document}
\begin{center}
  \section*{Worksheet 3: Random Permutations}
\end{center}
\begin{problem}
  Suppose several distinct playing cards are laid out in a line like this:
  \begin{center}
    \crdAs \crdtwos \crdtres \crdfours
  \end{center}
  A \demph{random transposition} is where two different cards are selected at random and
  their positions swapped. For example, the transposition $(23)$ swaps cards 2
  and 3. If the cards started in the above order, then the effect of applying
  $(23)$ would be to reorder the cards as follows:
  \begin{center}
    \crdAs \crdtres \crdtwos \crdfours
  \end{center}

  
  Anna's nemesis Mark challenges her to a betting game. In the game, Mark lays
  out four cards in the order $1,2,3,4$ and then performs $k$ random
  transpositions, one after the other. As this happens, Anna is blindfolded,
  so she can't see which transpositions were performed---however, because she
  can hear Mark moving the cards, she does know the value of $k$. Anna then
  has to guess the final order of the cards. If she guesses correctly, Mark
  will pay her $\$20$; otherwise she has to pay Mark $\$1$.

  Mark believes that as long as he performs at least $4$ transpositions, then
  all $4!=24$ permutations of the cards will be equally likely, so that Anna
  is expected to lose money when she plays the game. Moreover Mark believes
  that Anna is a sucker and he can make lots of money from her.

  But Mark is wrong. Design a betting strategy for Anna to bleed Mark for all
  he's worth.

  \Fl{Hint:}\textit{
  First consider the simpler problem where the game is played with
  3 cards. Write down a table which gives the probability mass function for
  each of the 6 permutations in that case:}
  
\begin{center}
  \begin{tabular}{>{\centering\arraybackslash}m{3cm} |>{\centering\arraybackslash}m{5cm}}
    \toprule
    \textbf{Permutation (in cycle notation)} & \textbf{Resulting Card Order} \\
    \midrule
    \Large(1)(2)(3)   & \scalebox{0.6}{\crdAs \crdtwos \crdtres} \\
    \Large(1)(23)     & \scalebox{0.6}{\crdAs \crdtres \crdtwos} \\
    \Large(12)(3)     & \scalebox{0.6}{\crdtwos \crdAs \crdtres} \\
    \Large(13)(2)     & \scalebox{0.6}{\crdtres \crdtwos \crdAs} \\
    \Large(123)       & \scalebox{0.6}{\crdtwos \crdtres \crdAs} \\
    \Large(132)       & \scalebox{0.6}{\crdtres \crdAs \crdtwos} \\
    \bottomrule
  \end{tabular}
\end{center}

\textit{Do this for each $k=2,3,4,5$. Once you understand the case with 3 cards,
  move to four.}

\end{problem}
\newpage
\textit{Note that in the next two problems, we will use the notation $AB$ to denote $A\cap B$.}


\begin{problem}[Inclusion-Exclusion]
  In this problem we will generalize the inclusion-exclusion principle
  \begin{equation}\label{eq:1}
    \P\left[A_{1}\cup A_{2} \right] = \P\left[A_{1} \right]+\P\left[A_{2} \right]-\P\left[A_{1}\cap A_{2} \right]
  \end{equation}
  to more than two sets. The formulas are a bit complicated, but we've broken
  it up into hopefully doable chunks.
  \begin{enumerate}[(a)]
    \setlength{\itemsep}{4em}
    \item \label{item:Ia} Let's start with some useful preliminaries. Show
    that $\E\left[\mathbf{1}_{A} \right]= \P\left[A \right]$ for any event
    $A$, and that $\mathbf{1}_{A} \mathbf{1}_{B} = \mathbf{1}_{AB}$ for any
    events $A$ and $B$.
    \item \label{item:Ib} The following polynomial equality holds for all $n \geq 1$:
    {\footnotesize
      \begin{align*}
      (1-x_{1})(1-x_{2})\cdots(1-x_{n})
      % &= \sum_{k=1}^{n} (-1)^{k} \left( \sum_{1 \leq i_{1}<i_{2}<\ldots i_{k}\leq n}a_{i_{1}}a_{i_{2}}\cdots a_{i_{k}}  \right) \\
        &= 1- \left(\sum_{1 \leq i \leq n} x_{i}\right) + \left(\sum_{1 \leq i<j \leq n} x_{i}x_{j}\right)-\left(\sum_{1 \leq i<j<k\leq n} x_{i}x_{j}x_{k}\right)+ \ldots+ (-1)^{n}\left(x_{1}x_{2}\cdots x_{n}\right) 
    \end{align*}}
    This formula can be proved by induction, but for now let's just convince
    ourselves that it holds by verifying the case when $n=4$.
    \item \label{item:Ic} Let $A_{1},\ldots,A_{n}$ be events. Show that
    \begin{equation*}
      \mathbf{1}_{A_{1}^{c}\cdots A_{n}^{c}} = (1-\mathbf{1}_{A_{1}})(1-\mathbf{1}_{A_{2}})\cdots(1-\mathbf{1}_{A_{n}}).
    \end{equation*}
    \item \label{item:Id} Using parts \ref{item:Ia}, \ref{item:Ib}, and \ref{item:Ic}, prove the following formula:
    {\footnotesize\begin{align*}
      \P\left[A_{1}^{c} \cdots A_{n}^{c} \right] 
      &= 1- \left(\sum_{1 \leq i \leq n} \P\left[A_{i} \right]\right) + \left(\sum_{1 \leq i<j \leq n} \P\left[A_{i} A_{j} \right]\right)-\left(\sum_{1 \leq i<j<k\leq n} \P\left[A_{i}A_{j}A_{k} \right]\right) + \ldots+ (-1)^{n}\P\left[A_{1} \cdots A_{n} \right]
    \end{align*}}
    \item Deduce from the formula in \ref{item:Id} that
    {\footnotesize\begin{align*}
      \P\left[A_{1}\cup \cdots \cup A_{n} \right] 
      &=\left(\sum_{1 \leq i \leq n} \P\left[A_{i} \right]\right) - \left(\sum_{1 \leq i<j \leq n} \P\left[A_{i} A_{j} \right]\right) +\left(\sum_{1 \leq i<j<k\leq n} \P\left[A_{i}A_{j}A_{k} \right]\right) + \ldots+ (-1)^{n-1}\P\left[A_{1} \cdots A_{n} \right].
    \end{align*}}
  \item The equation from the previous part is called \demph{the
    inclusion-exclusion principle.} It can be written more compactly in the
  following form (you don't have to show this):
    \begin{equation}\label{eq:inclusion-exclusion}
      \P\left[A_{1}\cup \ldots \cup A_{n} \right] = \sum_{k=1}^{n} (-1)^{k-1} \left[ \sum_{I\subseteq \left\{1,\ldots,n\right\}: |I|=k} \P\left[\cap_{i\in I}A_{i} \right]  \right] 
    \end{equation}
    To explain this equation, note that for each $k$, the summation in the
    square brackets sums over all subsets I of $\left\{1,\ldots,n\right\}$ of
    size $k$. (Meaning it sums over ${n\choose k}$ terms.) Use the
    inclusion-exclusion to write out a formula for
    \begin{equation*}
      \P\left[A\cup B\cup C \right]
    \end{equation*}
    and draw a venn diagram and interpret your formula geometrically.
  \end{enumerate}
\end{problem}

\newpage
\begin{problem}
  In this problem we consider the problem of a random permutation of $n$
  distinct elements, denoted $1,2,\ldots,n$. Let $N$ be the number of fixed
  points of the permutation. We will show that
  \begin{equation*}
    \P\left[N=0 \right] = \frac{1}{2!}-\frac{1}{3!}+\frac{1}{4!}-\ldots+(-1)^{n}\frac{1}{n!}
  \end{equation*}
  which implies that $\P\left[N=0 \right] \approx \frac{1}{e} \approx .37$ when $n$ is
  large.

  \textit{This problem uses the result of problem 2.}

  \begin{enumerate}[(a)]
    \setlength{\itemsep}{4em}

    \item \label{item:IIa}Let $A_{i}$ be the event that $i$ is a fixed point. Using the ideas
    of conditional probability, show that
    $\P\left[A_{i} \right] = \frac{1}{n}$, \quad
    $\P\left[A_{i}A_{j} \right]=\frac{1}{n(n-1)}$,\quad
    $\P\left[A_{i}A_{j}A_{k} \right]=\frac{1}{n(n-1)(n-2)}$, and so forth, so
    that in general, we have
    \begin{equation*}
      \P\left[\cap_{i\in I}A_{i} \right] = \frac{1}{n(n-1)(n-2)\cdots (n-k+2)}
    \end{equation*}
    whenever $I\subseteq \left\{1,2\ldots,n\right\}$ is a subset of size $k \geq 1$.
    
    \item \label{item:IIb} Let $k$ be a positive integer. How many terms are in the sum
    \begin{equation*}
      \sum_{I\subseteq \left\{1,\ldots,n\right\}: |I|=k} \P\left[\cap_{i\in I}A_{i} \right] 
    \end{equation*}
    \item \label{item:IIc}Using your answers to parts \ref{item:IIa} and \ref{item:IIb}, show that
    \begin{equation*}
      \sum_{I\subseteq \left\{1,\ldots,n\right\}: |I|=k} \P\left[\cap_{i\in I}A_{i} \right]  = \frac{1}{k!}
    \end{equation*}
    for each $k=1,2,\ldots$.
    \item Plugging \ref{item:IIc} into \Cref{eq:inclusion-exclusion}, deduce
    that
    \begin{equation*}
      \P\left[A_{1}\cup \cdots\cup A_{n} \right] = 1 - \frac{1}{2!}+\frac{1}{3!}-\frac{1}{4!}+\ldots+(-1)^{n-1}\frac{1}{n!}
    \end{equation*}
    \item Using the formula $e^{x} = \sum_{k=0}^{\infty}\frac{x^{k}}{k!}$,
    deduce that
    \begin{equation*}
      \P\left[A_{1}\cup \cdots\cup A_{n} \right] \approx 1-\frac{1}{e}
    \end{equation*}
    and hence that
    \begin{equation*}
      \P\left[A_{1}^{c}\cap A_{2}^{c}\cap\cdots \cap A_{n}^{c} \right] \approx \frac{1}{e}
    \end{equation*}
    \item Interpret this result in words.
    \item How close is this approximation when $n=4$? What about $n=5$?
  \end{enumerate}
\end{problem}


\newpage
\begin{problem}[Random Permutations II]

  In this problem, we will show that the expected number of cycles in a random
  permutation of $n$ elements is approximately $\log(n)$.

  \begin{enumerate}[(a)]
    \setlength{\itemsep}{4em}

    \item Let $i,j\in \left\{1,\ldots,n\right\}$ such that $i\neq j$. Let
    $C_{ij}$ be the event that $i$ and $i$ form a cycle of order 2. What is $\P\left[C_{ij} \right]$?

    \item Let $N_{2}$ be the number of cycles of order 2. Explain why
    \begin{equation*}
      N_{2} = \sum_{1 \leq i<j \leq n} \mathbf{1}_{C_{ij}}.
    \end{equation*}
    How many terms are in this sum?
    \item Show that $\E\left[N_{2} \right]=\frac{1}{2}$?
    \item \label{item:1} For each
    $k\in \left\{1,2,\ldots,n\right\}$, let $N_{k}$ be the number of cycles of
    order $k$. Using similar arguments as in the previous part of this
    problem, one can show that
    \begin{equation*}
      \E\left[N_{k} \right] = \frac{1}{k}.
    \end{equation*}
    \item Deduce from \ref{item:1} that the expected number of cycles in a random
    permutation of $n$ elements is the \demph{$n^{\rm th}$ harmonic number}
    $H_{n}$, defined as
    \begin{equation*}
      H_{n} = 1+\frac{1}{2}+\frac{1}{3}+\ldots+\frac{1}{n}.
    \end{equation*}
    \item  Conclude that as $n \to \infty$, the expected number of cycles grows
    at a rate proportional to $\log(n)$ by showing that
    \begin{equation*}
      \lim_{n\to\infty} \frac{\log(n) }{H_{n}} = 1.
    \end{equation*}
    \item If you shuffle a deck of 52 playing cards, the resulting rearrangment of the
    cards corresponds to some permutation of the set
    $\left\{1,2,\ldots,52\right\}$. What is a good guess for the number of
    cyles in that permutation?
  \end{enumerate}
\end{problem}

\end{document}