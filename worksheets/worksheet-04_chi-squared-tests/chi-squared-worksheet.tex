
%%% THIS DOCUMENT NEEDS TO BE COMPILED WITH LUATEX OR LUALATEX ENGINE

\documentclass[10pt]{amsart}
% Math Packages
\usepackage{amsmath, mathtools}
\usepackage{amssymb}
\usepackage{amsthm}
\usepackage{amsfonts}
\usepackage{bbm}
\usepackage{breqn}
\usepackage[margin=1in]{geometry}
\usepackage{graphicx}
\usepackage{tikz}
\usetikzlibrary{arrows.meta}
\usetikzlibrary{calc}
\usepackage{forest}
\usepackage{tikz-qtree}
\graphicspath{ {./images/} }
\usepackage{hyperref}
\usepackage[capitalize]{cleveref}
\usepackage[shortlabels]{enumitem}
\usetikzlibrary{arrows,matrix,positioning}
\usepackage{multicol}

% for the pipe symbol
\usepackage[T1]{fontenc}

% Citing theorems by name. (source: https://tex.stackexchange.com/questions/109843/cleveref-and-named-theorems)
\makeatletter
\newcommand{\ncref}[1]{\cref{#1}\mynameref{#1}{\csname r@#1\endcsname}}

\def\mynameref#1#2{%
  \begingroup
  \edef\@mytxt{#2}%
  \edef\@mytst{\expandafter\@thirdoffive\@mytxt}%
  \ifx\@mytst\empty\else
  \space(\nameref{#1})\fi
  \endgroup
}
\makeatother

% Colorful Notes
\usepackage{color} \definecolor{Red}{rgb}{1,0,0} \definecolor{Blue}{rgb}{0,0,1}
\definecolor{Purple}{rgb}{.5,0,.5} \def\red{\color{Red}} \def\blue{\color{Blue}}
\def\gray{\color{gray}} \def\purple{\color{Purple}}
\newcommand{\rnote}[1]{{\red [#1]}} % \rnote{foo} gives '[foo]' in red
\newcommand{\pnote}[1]{{\purple [#1]}} % \pnote{foo} gives '[foo]' in purple
\newcommand{\bnote}[1]{{\blue #1}} % \bnote{foo} gives 'foo' in blue
\newcommand{\gnote}[1]{{\gray #1}} % \gnote{foo} gives 'foo' in gray
\newcommand{\Max}[1]{{\purple [#1]}} % \bnote{foo} then 'foo' is blue


% Claim numbering (the counter restarts after each proof environment)
\newcounter{claimcount}
\setcounter{claimcount}{0}
\newenvironment{claim}{\refstepcounter{claimcount}\par\addvspace{\medskipamount}\noindent\textbf{Claim \arabic{claimcount}:}}{}
\usepackage{etoolbox}
\AtBeginEnvironment{proof}{\setcounter{claimcount}{0}}
\newenvironment{claimproof}{\par\addvspace{\medskipamount}\noindent\textit{Proof of Claim  \arabic{claimcount}.}}{\hfill\ensuremath{\qedsymbol} \tiny{Claim}

  \medskip}
% Add claim support to cleverref
\crefname{claimcount}{Claim}{Claims}


% Math Environments
\newtheorem{theorem}{Theorem}
\newtheorem{assumption}[theorem]{Assumption}
\newtheorem{lemma}[theorem]{Lemma}
\newtheorem{proposition}[theorem]{Proposition}
\newtheorem{corollary}[theorem]{Corollary}
\newtheorem{question}[theorem]{Question}
\theoremstyle{definition}
\newtheorem{definition}[theorem]{Definition}
\newtheorem{remark}[theorem]{Remark}
\newtheorem{example}[theorem]{Example}
\newtheorem{notation}[theorem]{Notation}
\newtheorem{problem}[theorem]{Problem}

% Matrices and Column Vectors. 
\usepackage{stackengine}
\setstackgap{L}{1.0\normalbaselineskip}
\usepackage{tabstackengine}
\setstackEOL{;}% row separator
\setstackTAB{,}% column separator
\setstacktabbedgap{1ex}% inter-column gap
\setstackgap{L}{1.5\normalbaselineskip}% inter-row baselineskip
\let\nmatrix\bracketMatrixstack  %Usage: \nmatrix{1,2,3\4,5,6}
\newcommand\cv[1]{\setstackEOL{,}\bracketMatrixstack{#1}} %usage: \cv{1,2,3}

% table alignment
\usepackage{array}
\usepackage{booktabs}

% Custom Math Coqmmands
\newcommand{\vt}{\vskip 5mm} % vertical space
\newcommand{\fl}{\noindent\textbf} % first line
\newcommand{\Fl}{\vt\noindent\textbf} % first line with space above
\newcommand{\norm}[1]{\left\lVert#1\right\rVert} % norm
\newcommand{\pnorm}[1]{\left\lVert#1\right\rVert_p} % p-norm
\newcommand{\qnorm}[1]{\left\lVert#1\right\rVert_q} % q-norm
\newcommand{\1}[1]{\textbf{1}_{\left[#1\right]}} % indicator function 
\def\limn{\lim_{n\to\infty}} % shortcut for lim as n-> infinity
\def\sumn{\sum_{n=1}^{\infty}} % shortcut for sum from n=1 to infinity
\def\sumkn{\sum_{k=1}^{n}} % shortcut for sum from k=1 to n
\def\sumin{\sum_{i=1}^{n}} % shortcut for sum from i=1 to n
\def\SAs{\sigma\text{-algebras}} % shortcut for $\sigma$-algebras
\def\SA{\sigma\text{-algebra}} % shortcut for $\sigma$-algebra
\def\Ft{\mathcal{F}_t} % time-indexed sigma-algebra (t)
\def\Fs{\mathcal{F}_s} % time-indexed sigma-algebra (s)
\def\F{\mathcal{F}} % sigma-algebra
\def\G{\mathcal{G}} % sigma-algebra
\def\R{\mathbb{R}} % Real numbers
\def\N{\mathbb{N}} % Natural numbers
\def\Z{\mathbb{Z}} % Integers
\def\E{\mathbb{E}} % Expectation
\def\P{\mathbb{P}} % Probability
\def\Q{\mathbb{Q}} % Q probability
\def\dist{\text{dist}} %Text 'dist' for things like 'dist(x,y)'
\newcommand{\indep}{\perp \!\!\! \perp}  %independence symbol
\def\Var{\mathrm{Var}} % Variance
\def\tr{\mathrm{tr}} % trace

% Brackets and Parentheses
\def\[{\left [}
    \def\]{\right ]}
% \def\({\left (}
%   \def\){\right )}



\usepackage{color}
\definecolor{Red}{rgb}{1,0,0}
\definecolor{Blue}{rgb}{0,0,1}
\definecolor{Purple}{rgb}{.5,0,.5}
\def\red{\color{Red}}
\def\blue{\color{Blue}}
\def\gray{\color{gray}}
\def\purple{\color{Purple}}
\definecolor{RoyalBlue}{cmyk}{1, 0.50, 0, 0}
\newcommand{\dempfcolor}[1]{{\color{RoyalBlue}#1}} 
\newcommand{\demph}[1]{\dempfcolor{{\sl #1}}}

% comment exactly one of the following line to show / hide the solutions
% \newcommand{\solution}[1]{{\purple #1}} % uncomment to show the solutions
\newcommand{\solution}[1]{} % uncomment to hide the solutions

\usepackage{pst-poker}


\begin{document}
\begin{center}
  \section*{Worksheet 4: Chi-Squared Tests}
\end{center}
\Fl{Useful Fact:} The pdf of a chi-squared random variable with $d$ degrees of
freedom is
\begin{equation*}
  f(x) = \frac{1}{2^{d/2}\Gamma(d/2)}x^{\frac{d}{2}-1}e^{-x^{2}/2}, \quad x>0.
\end{equation*}

\Fl{Goodness-of-Fit Test}

Suppose we have a multinomial random variable with $k$ outcomes, labelled
$1,\ldots,k$, constructed from $n$ trials such that in each trial outcome $i$
occurs with probability $p_{i}$. The \demph{chi-squared statistic} is the
quantity
\begin{equation*}
  \chi^{2}:= \sum_{i=1}^{k} \frac{\left(N_{i}-E_{i}\right)^{2}}{E_{i}} 
\end{equation*}
where $E_{i}$ is the expected number of times outcome $i$ occurs, and $N_{i}$
is the number of times outcome $i$ is observed. For a goodness of fit test
using this statistic, $\chi^{2}$ has a chi-squared distribution with $k-1$
degrees of freedom.


\begin{problem}[Goodness-of-fit test]
  According to the Mars Wrigley Confectionery Company, the color distribution
  of a bag of skittles is skittles, the null hypothesis
  \begin{equation*}
    H_{0}: p_{\rm green} = p_{\rm yellow}=p_{\rm red} = p_{\rm orange} = p_{\rm purple} = 0.2
  \end{equation*}
  the alternative hypothesis is
  \begin{equation*}
    H_{1}: \text{at least one of the probabilities is different from $0.2$}
  \end{equation*}

  \begin{enumerate}[(a)]
    \setlength{\itemsep}{4em}
    \item Assuming that $H_{0}$ is true, what is $E_{i}$ for all $i\in
    \left\{\text{green}, \text{yellow}, \text{red}, \text{orange}, \text{purple}\right\}$?
    \item Fill in the following table with expected and observed counts:
    \begin{center}
      \begin{tabular}{|l|l|l|l|l|l|}
        \hline
        & \textbf{green} & \textbf{yellow} & \textbf{  red  } & \textbf{orange} & \textbf{purple} \\ \hline
        \textbf{expected} &  $\phantom{\displaystyle\int}$ &   &       &   &   \\ \hline
        \textbf{observed} &  $\phantom{\displaystyle\int}$ &   &       &   &   \\ \hline
      \end{tabular}
    \end{center}
    \vspace{-3em}
    \item Compute the chi-squared statistic $\chi^{2}$ for your bag of
    skittles
    \item Circle the correct phrase to complete the sentence: \textit{A larger
    chi-square value means the observed data is (very different from / very
    similar to) the expected data.}
    \item From theorem stated in class, $\chi^{2}$ has a chi-squared
    distribution with $k-1$ degrees of freedom. Set up and evaluate an
    integral to determine the p-value $\P\left[\chi^{2}\geq u\right]$, where
    $u$ is the value you got in the previous part of the problem.
    \item Write a sentence or two interpreting your result
  \end{enumerate}
\end{problem}

\newpage
\Fl{Test of Independence}
\begin{problem}[Test of Independence]
  A 2007 study of $n=186$ Japanese childen examined a possible relationship
  between the onset of autism-related developmental regression and exposure to
  the measles mumps rubella (MMR) vaccine.\footnote{Uchiyama et al, \textit{J
      Autism Dev Disord}, 2007.
    \url{https://doi.org/10.1007/s10803-006-0157-3}} One of their analyeses
  involved the following dataset: \Fl{Table of observed values}
  \begin{center}
    \begin{tabular}{l|l|l|}
      & \textbf{Vaccinated} & \textbf{Not Vaccinated} \\ \hline
      \textbf{Autism}    & $15\phantom{\displaystyle\int}$ & $39$ \\ \hline
      \textbf{No Autism} & $45\phantom{\displaystyle\int}$ & $87$ \\ \hline
    \end{tabular}
  \end{center}
  \vspace{1em}

  \begin{enumerate}[(a)]
    \setlength{\itemsep}{3em}
    \item Find the following values
    \begin{itemize}
      \item total number of children exhibiting autistic regression
      \item total number of children not exhibiting autistic regression
      \item total number of vaccinated children
      \item total number of unvaccinated children
    \end{itemize}
    \vspace{-2em}
    \item Let $p$ be the probability that a child exhibits autistic
    regression, and let $q$ be the probability that a child is vaccinated. Use
    your answers from the previous part to estimate $p$ and $q$.

    \item Under the assumption that autism and vaccination are independent, we
    can compute expected values for each of the four categories of outcomes,
    which we'll call $e_{11},e_{12},e_{21},$ and $e_{22}$, and arrang them in
    the following table:
    
    \Fl{Table of expected counts}
    \begin{center}
      \begin{tabular}{l|l|l|}
        & \textbf{Vaccinated} & \textbf{Not Vaccinated} \\ \hline
        \textbf{Autism}    & $e_{11}=\phantom{\displaystyle\int} $ & $e_{12}=$ \\ \hline
        \textbf{No Autism} & $e_{21}=\phantom{\displaystyle\int}$ & $e_{22}=$ \\ \hline
      \end{tabular}
      \vspace{1em}
    \end{center}
    
    For example, under this assumption, the expected number of children who
    are both vaccinated and exhibit autistic regression is
    \begin{equation*}
      e_{11} = npq
    \end{equation*}
    Fill in the values of the table.
    \vspace{-2em}
    \item Compute the chi-squared statistic
    \begin{equation*}
      \chi^{2} = \sum_{\text{all }i,j} \frac{\left( o_{ij}-e_{ij} \right)^{2} }{e_{ij}} 
    \end{equation*}
    where $o_{11},o_{12},o_{21},$ and $o_{22}$ are the observed values. 
    \item Set up and evaluate an integral to compute a $p$-value for your
    $\chi^{2}$ statistic.
    \item Write a sentence or two interpreting your result.
  \end{enumerate}
\end{problem}


\newpage
\begin{problem}[Simpson's Paradox: The Perils of Aggregation]

  In this problem we compare the performance of two medical centers, Hospital
  A and Hospital B. We have the following survival data for surgery patients from
  these two hospitals from the last 6 weeks:
  \begin{center}
    \vspace{1em}
    \begin{tabular}{lcc}
      \hline
      \textbf{Outcome} & \textbf{Hospital A} & \textbf{Hospital B} \\
      \hline
      Died     &   63     & 16       \\
      Survived & 2037       & 784       \\
      \hline
    \end{tabular}
    \vspace{2em}
  \end{center}
  \begin{enumerate}[(a)]
    \setlength{\itemsep}{2.5em}
    \item \label{item:1} Compare the survival rates of the two hospitals: what percentage of
    patients at Hospital A survive? What about Hospital B?
    \item Is there a relationship between patient survival and which hospital
    they go to? Do a chi-squared test of independence. What $p$-value do you
    get? In one or two sentences, state your null hypothesis and draw a
    conclusion about it.
    \item We now consider a third factor: the condition of the patient when
    they are admitted for surgery. Patients are either classified as ``poor
    condition'' or ``good condition''. Here's the more detailed data:

    \begin{center}
      \begin{tabular}{cc}
        \begin{tabular}{lcc}
          \multicolumn{3}{c}{\textbf{Good Condition}} \\ \hline
          \textbf{Outcome} & \textbf{Hospital A} & \textbf{Hospital B} \\ \hline
          Died     &    6    & 8       \\
          Survived & 594       & 592       \\ \hline
        \end{tabular}
                       \quad \quad \quad \quad    &
                             \begin{tabular}{lcc}
                               \multicolumn{3}{c}{\textbf{Poor Condition}} \\ \hline
                               \textbf{Outcome} & \textbf{Hospital A} & \textbf{Hospital B} \\ \hline
                               Died     &  57      & 8       \\
                               Survived & 1443       & 192       \\ \hline
                             \end{tabular}
      \end{tabular}
    \end{center}
    \item What fraction of Hospital A's patients were classified as good
    condition? What about Hospital B?
    \item \label{item:2} Out of the patients in \textbf{good condition} who went to hospital A,
    what fraction survived? What about hostpial B? Compare these two survival
    rates.
    \item \label{item:3} Out of the patients in \textbf{bad condition} who went to hospital A,
    what fraction survived? What about hostpial B? Compare these two survival
    rates. 
    \item Compare your answers to parts
    \ref{item:1}~\ref{item:2}~\ref{item:3}. What you are observing is called
    \demph{Simpson's Paradox}. How can this discrepancy be
    explained? If you are facing surgery, should you go to Hospital A or
    Hospital B?
    
    \item Fill in the following contingency table:
    \begin{center}
      \vspace{1em}
      \begin{tabular}{|l|c|c|}
        \hline
        \textbf{Outcome} & \textbf{Good Condition} & \textbf{Bad Condition} \\
        \hline
        Died     &    $\phantom{\displaystyle\int}$    &        \\
        \hline
        Survived &        &        $\phantom{\displaystyle\int}$\\ 
        \hline
      \end{tabular}
    \end{center}
    \item Is there a reslationship between patient survival and the condition
    that they are admitted for surgery. Perform a chi-squared test of
    independence and report a $p$-value.
  \end{enumerate}
\end{problem}





\begin{problem}[Another example of Simpson's Paradox]
  The influence of race on imposition of the death penalty for murder has been
  much studied and contested in the courts. The following three-way table
  classifies 326 cases in which the defendant was conficted of
  murder.\footnote{Data from M. Radelet, \textit{American Sociological Review}, 1981.
    \url{https://doi.org/10.2307/2095088}} The three varaibles are the
  defendant's race, the victim's race, and whether or not hte defendant was
  sentenced to death.

  \begin{center}
    \begin{tabular}{cc}
      \begin{tabular}{lcc}
        \multicolumn{3}{c}{\textbf{White Defendants}} \\ \hline
        \textbf{Victim Race} & \textbf{Yes} & \textbf{No} \\ \hline
        White & 19 & 132 \\
        Black & 0  & 9   \\ \hline
      \end{tabular}
                            \quad \quad \quad \quad &
                               \begin{tabular}{lcc}
                                 \multicolumn{3}{c}{\textbf{Black Defendants}} \\ \hline
                                 \textbf{Victim Race} & \textbf{Yes} & \textbf{No} \\ \hline
                                 White & 11 & 52  \\
                                 Black & 6  & 97  \\ \hline
                               \end{tabular}
    \end{tabular}
  \end{center}
  \vspace{3em}
  \begin{enumerate}[(a)]
    \setlength{\itemsep}{7em}
    \item From these data make a 2-way table of defendant's race by death
    penalty. Does it look like there is a relationship between these two
    things? 
    \item Suppose you performed a chi-squared test of independence and for
    the table you just made. You get $p=.64$. How do you interpret this
    result?
    \item Show that Simpson's paradox holds: specifically, that
    \begin{itemize}
      \item a higher percent of white defendants are sentenced to death
      overall
      \item but for both black and white victims, a higher percent of black
      defendants are sentenced to death.
    \end{itemize}
    \item Basing your reasoning on the data, explain why the paradox holds
    in language that a judge could understand.
  \end{enumerate}


 
\end{problem}


\newpage
\begin{problem}[One True Love]
  
  A poll is taken of $n=2625$ individuals; people were survedy on some
  demographic information and also asked whether or not they believed in ``the
  one true love''.

  In this problem, we test whether or not there is a relationshp between the
  following two factors
  \begin{itemize}
    \item Factor 1: education level
    \item Factor 2: sentiment about true love
  \end{itemize}

  We can construct a two-way contingency table for these two factors, as follows:
  \begin{center}
    \begin{tabular}{l|r|r|r|r}
      \textbf{Response} & \textbf{HS} & \textbf{Some College} & \textbf{College} & \textbf{Total} \\
      \hline
      Agree       & 363 & 176 & 196 & 735 \\
      Disagree    & 557 & 466 & 789 & 1812 \\
      Don't Know  &  20 &  26 &  32 & 78 \\
      \hline
      \textbf{Total} & 940 & 668 & 1017 & 2625 \\
    \end{tabular}
  \end{center}
  This contingency table has $r=3$ rows and $c=3$ columns (plus a row and
  column for totals). The data counts themselves take the form of a $3\times 3$ matrix
  \begin{equation*}
    \begin{matrix}
      Y_{11} & Y_{12} & Y_{13} & \cdots & Y_{1c}\\
      Y_{21} & Y_{22} & Y_{23} & \cdots & Y_{2c}\\
      \vdots &&&&\vdots \\
      Y_{r1} & Y_{r2} & Y_{r3} & \cdots & Y_{rc}\\
    \end{matrix}
  \end{equation*}
  where $Y_{ij}$ is the number of samples for which factor $1$ is $i$ and
  factor 2 is $j$. We've also added row and column totals in the margins. We
  can construct a coresponding table of ``estimated expected values'' of the form

  \begin{equation*}
    \begin{matrix}
      E_{11} & E_{12} & E_{13} & \cdots & E_{1c}\\
      E_{21} & E_{22} & E_{23} & \cdots & E_{2c}\\
      \vdots &&&&\vdots \\
      E_{r1} & E_{r2} & E_{r3} & \cdots & E_{rc}\\
    \end{matrix}
  \end{equation*}
  by the formula
  \begin{equation}\label{eq:1}
    E_{ij} = n\times \frac{(\text{sum of row }i)}{n} \times \frac{(\text{sum of column }j)}{n}
  \end{equation}
  which gives us a new table of estimated expected values.

  \Fl{Fact:} For contingency tables with $r$ rows and $c$ columns, the chi-squared
  statistic
  \begin{equation*}
    \chi^{2} := \sum_{i=1}^{r}\sum_{j=1}^{c} \frac{\left( Y_{ij}-E_{ij} \right)^{2} }{E_{ij}}  
  \end{equation*}
  has chi-squared distribution with $(r-1)\times(c-1)$ degrees of freedom.

  \fl{Questions:}
  \begin{enumerate}[(a)]
    \setlength{\itemsep}{3em}
    \item Explain why the formula in \Cref{eq:1} holds.
    \item What is the degrees of freedom for the chi-squared statistic in this
    problem? Write down the formula for the pdf of the chi-squared statistic.
    \item Do the $\chi^{2}$ test of independence (i.e., compute a table of
    expected values, find $\chi^{2}$, and use the known distribution of the
    statistic to compute a $p$-value). 
    \item Intepret your result.
  \end{enumerate}
\end{problem}
\end{document}